\documentclass[a4paper, 11pt]{article}
\usepackage{times}
\usepackage[left=2cm, top=3cm, text={17cm, 24cm}]{geometry}
\usepackage[utf8]{inputenc}
\usepackage[czech]{babel}
\usepackage{times}
\usepackage{url}

\providecommand{\uv}[1]{\quotedblbase#1\textquotedblleft}

\DeclareUrlCommand\url{\def\UrlLeft{<}\def\UrlRight{>} \urlstyle{tt}}

\author{Ondřej Valeš}
\date{}

\begin{document}
	\begin{titlepage}
		\centering
		{\textsc{\Huge Vysoké učení technické v~Brně\\ \medskip
		\huge Fakulta informačních technologií}}
		
		\vspace{\stretch{0.382}}
		{\LARGE Typografie a publikování\,--\,3. projekt\\ \medskip
		\Huge Citace}
		
		\vspace{\stretch{0.618}}	 
		\Large \today \hfill Ondřej Valeš
	\end{titlepage}

\section{Typometrické systémy}

Typografické míry se zásadně liší od běžných metrických měr, protože jejich vzik si vyžádala potřeba přípravy tiskovin předtím, než.
Rozlišují se dva základní, nejpoužívanější systémy. Těmito systémy jsou \emph{Didotův} měrný systém, jehož základem jsou body a platí $1\,\mbox{b} = 0.03759\,\mbox{mm}$, a měrný 
systém \emph{pica} \cite{Sirucek:2006}.

Jednotky těchto systémů se potom používají pro definice velikostí mezer a znaků (důležité zejména pro správné strukturování dokumentů \cite{Rehorova:2006}). 
Další důležitou jednotkou je \emph{čtverčík}, jedná se čtverec o straně o velikosti stejné, jako je velikost písma \cite{printing_type:2004}.

\section{Zásady psaní mezer a pomlček}
\subsection{Mezery}

Velikost základní mezislovní mezery je potom třetina čtverčíku \cite{Beran:1994}. Speciálním typem mezery je nezlomitelná mezera, 
píše se u jednopísmenných předložek a mezi číslem a jednotkou. Nezlomitelnou mezeru je vhodné použít k oddělení skupin číslic (po třech) \cite{Polacek:2008a}. 
V ptostředí \LaTeX se k vysázení nezlomitelné mezery používá znak \verb|~| \cite{Lamport:1994}.

Dalším případem použití nezlomitelné mezery je zápis data a času (zde pozor na odlišnosti české a anglické typografie\,--\,používáme 24hodinové intervaly a 
časové údaje oddělujeme tečkou \cite{Polacek:2008b}).

\subsection{Pomlčky}

\begin{quote}
Jednou z věcí, kterými se odlišuje knižní sazba od strojopisu, je pomlčka. Zatímco se na psacím stroji používá pro spojovník a pomlčku stejný znak, 
ve fontech knižního písma nalezneme kromě spojovníku dokonce dvě pomlčky. Jednaje  dlouhá  (čtverčíková) a  druhá  je  krátká  (půlčtverčíková) \cite{Brezina:2006}.
\end{quote}

Pomlčka se v  prostředí \LaTeX napíše jako dvě pomlčky za sebou a je oddělena zůženou mezerou: \verb|\,--\,|. Používá se pro výrazné oddělení části textu.
Tedy tam kde je potřeba naznačit větší přestávku v řeči \cite{Dostal:1994}.

\newpage
\renewcommand{\refname}{Použitá literatura}
\bibliography{proj4}
\bibliographystyle{czechiso}

\end{document}